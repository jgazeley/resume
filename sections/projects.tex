\section{Projects}
\vspace{0.1cm}
% --- N64 Cartridge Reader ---
\noindent 
\textbf{N64 Cartridge Reader \& Save Manager} \quad \href{https://github.com/jgazeley/n64cartreader}{\faGithub \: github.com/jgazeley/n64cartreader}
\begin{highlightsforbulletentries}
    \item Managed the full lifecycle of a commercial product, taking the initial ATmega2560-based design from a custom PCB to manufacturing and selling over 100 units.
    % \item Currently engineering a V2, migrating the design to an RP2040 to add a user-friendly USB Mass Storage interface and reduce production cost by over 65\%.
\end{highlightsforbulletentries}

\vspace{0.2cm}
% --- 8-bit Hardware Arithmetic Unit ---
\noindent
\textbf{8-bit Hardware Arithmetic Unit (Adder/Subtractor)} \quad \href{https://github.com/jgazeley/light-sensor}{\faGithub \: github.com/jgazeley/light-sensor}
\begin{highlightsforbulletentries}
    \item Designed and built a mixed-signal hardware calculator from scratch, successfully integrating an analog front-end %(using an op-amp and ADC) 
    with a digital processing backend built from discrete %SN74LS
    logic gates.
\end{highlightsforbulletentries}

% \vspace{0.2cm}
% % --- Custom Automotive Embedded Control Modules ---
% \noindent
% \textbf{Custom Automotive Embedded Control Modules}
% \begin{highlightsforbulletentries}
%     \item Engineered a suite of embedded modules to control functions like remote start and an immobilizer bypass in a challenging 12V automotive environment.
%     % \item Designed and tested custom PCBs and firmware specifically optimized for reliability within a vehicle's electrical system.
% \end{highlightsforbulletentries}

% \vspace{0.2cm}
% % --- Automated Terrarium Controller ---
% \noindent
% \textbf{Automated Terrarium Controller} \quad \href{https://github.com/jgazeley/terrarium/blob/main/terrarium.ino}{\faGithub \: github.com/jgazeley/terrarium}
% \begin{highlightsforbulletentries}
%     \item Wrote Arduino firmware to monitor DHT22 temperature/humidity and light sensors, and to drive relays for heaters, fans, and grow-lights.
%     \item Implemented a PID loop in C++ to maintain target environmental conditions within ±1 °C/±5 \% relative humidity.
%     % \item Added timed light-cycle scheduling and manual override via push-buttons, with state feedback on an OLED display.
%     % \item Logged sensor readings and actuator events to an SD card for later analysis, and provided USB-serial debug output.
% \end{highlightsforbulletentries}
